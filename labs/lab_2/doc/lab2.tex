\documentclass[11pt, oneside]{article}   	% use "amsart" instead of "article" for AMSLaTeX format

\usepackage{float}

\usepackage{geometry}                		% See geometry.pdf to learn the layout options. There are lots.
\geometry{letterpaper}                   		% ... or a4paper or a5paper or ... 
%\geometry{landscape}                		% Activate for rotated page geometry
\usepackage[parfill]{parskip}    			% Activate to begin paragraphs with an empty line rather than an indent
\usepackage{graphicx}				% Use pdf, png, jpg, or eps§ with pdflatex; use eps in DVI mode
								% TeX will automatically convert eps --> pdf in pdflatex		
\usepackage{amssymb}
\usepackage{mathtools}
\usepackage{enumerate}
\usepackage{tikz}
\usepackage{listings}
\usepackage{color}
\definecolor{dkgreen}{rgb}{0,0.6,0}
\definecolor{gray}{rgb}{0.5,0.5,0.5}
\definecolor{mauve}{rgb}{0.58,0,0.82}
\lstset{frame=tb,
  language=C++,
  aboveskip=3mm,
  belowskip=3mm,
  showstringspaces=false,
  columns=flexible,
  basicstyle={\small\ttfamily},
  numbers=none,
  numberstyle=\tiny\color{gray},
  keywordstyle=\color{blue},
  commentstyle=\color{dkgreen},
  stringstyle=\color{mauve},
  breaklines=true,
  breakatwhitespace=true,
  tabsize=4
}

\usetikzlibrary{arrows}
%\usepackage[demo]{graphicx}
\usepackage{caption}
\usepackage{subcaption}
\def\firstcircle{(90:1.75cm) circle (2.5cm)}
\def\secondcircle{(210:1.75cm) circle (2.5cm)}
\def\thirdcircle{(330:1.75cm) circle (2.5cm)}

%SetFonts

%SetFonts
\begin{document}
\title{GPU Computing CSE 560 (Winter 2022) - Lab 2} 
\author{Aditya Singh Rathore - \texttt{2018007}}
\date{February 4, 2022}							% Activate to display a given date or no date


\maketitle

\section*{Constant Memory : Task}

\begin{lstlisting}
// initialize1.cu
__global__ void initialize1(float* C){
    int i = blockDim.x*blockIdx.x + threadIdx.x;
    if(i < LENGTH){
        C[i] = A[i] + B[i];
    }
}
//...
\end{lstlisting}

\begin{lstlisting}
// initialize2.cu
__global__ void initialize2(float *C){
    int i = blockDim.x*blockIdx.x + threadIdx.x;
    if(i < LENGTH){
        C[i] = A[blockIdx.x] + B[blockIdx.x];
    }
//...
\end{lstlisting}
\begin{tabular}{ |p{3cm}||p{3cm}|  }
 \hline
 \multicolumn{2}{|c|}{Code was tested on Nvidia 1050 Ti} \\
 \hline
 Code & Execution Time\\
 \hline
 initialize1.cu   & 0.00989895 ms\\
 initialize2.cu   & 0.00919106 ms\\
 \hline
\end{tabular} \\
\textbf{initialize2.cu is faster. }\\
The constant memory is cached. In \emph{initialize2.cu}, all threads in the warp will be reading same A[blockIdx.x] and B[blockIdx.x] as blockIdx.x is same for all of them, which is available in cache. 
\\
In \emph{initialize1.cu}, each threads will be reading different A[i] and B[i] which will lead to cache misses and lead comparatively larger execution time.
\end{document}  